\documentclass[12pt,oneside]{article}
\usepackage[T1]{fontenc}
\usepackage{latexsym}
\usepackage[activeacute,spanish]{babel}
\usepackage{amsfonts}
\usepackage{amsmath}
\usepackage{amssymb}
\usepackage{amsthm}
\usepackage{graphicx}
\usepackage[all]{xy}
\usepackage{tikz}
\usepackage[normalem]{ulem}
\usepackage{cancel}
\usepackage{soul}
\usepackage[retainorgcmds]{IEEEtrantools}
\usepackage{mathrsfs}
\usepackage{makeidx}
\newtheorem{prob}{Problema}
\addtolength{\hoffset}{-2cm}
\addtolength{\textwidth}{4cm}
\addtolength{\voffset}{-2.5cm}
\addtolength{\textheight}{5cm}
\pagestyle{empty}


\begin{document}


\begin{flushright}
{\large\textbf{Cova} \textnormal{Pacheco}, \textnormal{Felipe de Jes\'us}}
\end{flushright}



\begin{prob}[Ejerc\'icio 7.4.2] %[Fuente, capítulo, ejercicio.]
Es cada una de las siguientes relaciones antisim\'etrica, un orden parcial y/o un orden total?\\
(1) Sea $F$ un conjunto de personas en Francia, y sea $M$ la relaci\'on definida por $x M y$ si y solo si $x$ come m\'as queso anualmente que $y$, para todo $x,y \in F$.\\
(2) Sea $W$ el conjunto de todas las personas que alguna vez vivieron y alguna vez vivir\'an, y sea $A$ la relaci\'on en $W$ definida por $x A y$ si y solo si $y$ es ancestro de $x$  o si $y = x$, para todo $x,y \in W$.\\
(3) Sea $T$ el conjunto de todos los tri\'�ngulos en el plano, y sea $L$ la relaci\'on en $T$ definida por $s L t$ si y solo si $s$ tiene area menor o igual a $t$, para cualquier tri\'angulo en $s,t \in T$.\\
(4) Sea $U$ el conjunto de ciudadanos actuales en E.U.A., y sea $Z$ la relaci\'on en $U$ definida por $xZy$ si y solo si el N\'umero de Seguro Social de $x$ es mayor que el N\'umero de Seguro Social de $y$ para toda $x,y \in U$.
\end{prob}

\begin{proof}
(1)\\
$xMy$ si y solo si $x$ come m\'as queso al a�o que $y$\\
$xMx$ $x$ come m\'as queso al a�o que $x$   NO.\\
$\Longrightarrow$ No es Orden Parcial ni Orden Total (ref).\\\

(2)$ xAy$ si y solo si $x$ es ancestro de $y$ \'o $y = x$\\
$xAy$ entonces $x = x$ $\Longrightarrow$ es Reflexiva\\
$xAy$, $yAz$ entonces $x = y$ $\Longrightarrow$ Antisim\'etrica\\
$xAy$, $yAz$ entonces $xAz$ $\Longrightarrow$ Transitiva\\
	(a) $x$ es ancestro de $y$, $y$ es ancestro de $z$ $\Longrightarrow$ $xAz$\\
	(b) $x = z$\\
	(c) $x$ es ancestro de $y$, $y = z$ $\Longrightarrow$  $xAz$\\
	(d) T\\
		$\Longrightarrow$  OP (ref)\\
		$xAy$ \'o $yAx$\\
		No es OT\\\

(3) $sZt$ si y s\'olo si $As \leq At$\\
$sZs \Longrightarrow Reflexivo$\\
$sZt$, $tZs$ ent. $As \leq At$, $At \leq As$\\
Pero no es Antisim\'etrico\\
$\Longrightarrow$ no es Orden Parcial ni Orden Total.\\\

(4) $xZy$ si y s\'olo si nssde $x > $ nssy\\
$ xZx$ nssx $>$ nssx\\
$\Longrightarrow$ no es Orden Parcial (ref) ni Orden Total.\\\\\\\\\\\\\\\
	

\end{proof}

\begin{prob}[Ejerc\'icio 7.4.3] %[Fuente, capítulo, ejercicio.]
Sea $A \subset |N$ un subconjunto, y sea $\preceq$ la relaci\'on en $A$ definida por $a \preceq b$ si y solo si $b = a^k$ para alg\'una $k \in |N$, para todo $a,b \in A$.Probar que $(A,\preceq)$ es un COPO. Es $(A,\preceq)$ un conjunto totalmente ordenado?.
\end{prob}

\begin{proof}
Transitiva\\
$a \preceq b$, $b \preceq c$\\
$a^k = b$, $b^l = c$\\
$(a^k)^l = c$ $\Longrightarrow$ $a^(k*l)$ $= c$\\
como $k$ y $l \in N$\\
entonces $k*l \in N$\\
Por lo tanto $a \preceq c$\\\

Antisim\'etrica\\
$a \preceq b$ y $b \preceq a$\\
$a^k = b$, $b^l = a$\\
$(a^k)^l = a$ $\Longrightarrow$ $(b^k)^l = b$
$\Longrightarrow$ $k * l = 1$ $\Longrightarrow$ $k = 1$ y $l = 1$\\
Por lo tanto $a = b$.\\\

Reflexiva\\
$a^1 = a$ y $1 \in N$\\
$\Longrightarrow$ $a \preceq a$.

\end{proof}



\begin{prob}[Ejerc\'icio 7.4.5] %[Fuente, capítulo, ejercicio.]
(1) Da un ejemplo de una relaci\'on en $R$ que sea transitiva y antisim\'etrica pero no sim\'letrica ni reflexiva.\\
(2) Sea $A$ un conjunto no vac\'io, y sea $R$ una relaci\'on en $A$. Suponemos que $R$ es tanto sim\'etrica como antisim\'etrica. Probar que todo elemento de $A$ est\'a relacionado como m\'animo a s\'i mismo. 
\end{prob}

\begin{proof}
$A = \{ 1,2\}$		$(R,\le)$\\
$R \{(1,2)\}$\\
$aRb$ y $bRa$ pero $a \neq b$\\
$aRb$ y $bRc$ $\Longrightarrow$ $aRc$
\end{proof}



\begin{prob}[Ejerc\'icio 7.4.6] %[Fuente, capítulo, ejercicio.]
(1) Probar si el COPO tiene un elemento m\'as grande, entonces el elemento m\'as grande es \'�nico, y si un COPO tiene un elemento m\'as peque�o, entonces el elemento m\'as peque�o es \'�nico.\\
(2) Encuentra un ejemplo de un  COPO que tiene tanto un elemento m\'as peque�o como un elemento m\'as grande, un ejemplo que tiene un elemento m\'as peque�o, pero no uno m\'as grande, un ejemplo que tiene un elemento m\'as grande pero no un elemento m\'as chico y un ejemplo que no tiene ninguno.
\end{prob}

\begin{proof}
(1) $x$ es m\'aximo en $A$\\
$x \geq y$, $\forall y \in A$\\
Sup $X1, X2$ ambos m\'aximos en $A$\\
$X1 \geq X2$ y $X2 \geq X1$\\
$X1 = X2$.\\\

(2) A=$(1,2,3,4)$ $\leq$               B= N $\leq$
\\minimo = 1                             minimo 0 
\\maximo = 4                            maximo no tiene

Z(negativo)   $\leq$                 Z(positivo) $\leq$             
\\maximo = -1                           maximo no tiene
\\minimo no tiene                      minimo 0

\end{proof}



\begin{prob}[Ejerc\'icio 7.4.7] %[Fuente, capítulo, ejercicio.]
Probar que el elemento m\'as grande de un COPO es un elemento m\'aximo, y que un elemento m\'as peque�o de un COPO es un elemento minimo. 
\end{prob}

\begin{proof}

\end{proof}




\begin{prob}[Ejerc\'icio 7.4.13] %[Fuente, capítulo, ejercicio.]
Sea $(A, \preceq)$ un COPO, sea $X$ un conjunto y sea $h: X \longrightarrow A$ una funci\'on. Sea $\preceq'$ la relaci\'on en $X$ definida por $x \preceq' y$ si y solo si $h(x) \preceq h(y)$, para todo $x,y \in X$. Probar que $(X , \preceq')$ es un COPO.
\end{prob}

\begin{proof}
$(A,R) COPO$\\
$h: X \longrightarrow A$\\
$R'$ rel en $X$\\
$xR'y$ syss\\
$h(x) R h(y)$\\
$(x,R')$ COPO\\
$(A,R)$ $\forall x,y,z \in A$\\
(1) $xRx$\\
(2) $xRy$, $yRx$ ent. $x = y$\\
(3) $xRy$, $yRz$ ent. $xRz$\\\

$h(x) = d$, $x,y,z \in X$\\
$h(y) = e$, $d,e,k \in A$\\
$h(z) = k$\\
$xR'x$ $\Leftrightarrow$ $h(x)Rh(x)$ $\Leftrightarrow$ $dRd$ (Por (1))\\\

$xR'y$, $yR'x$ ent. $x=y$\\
$h(x)Rh(y)$, $h(y)Rh(x)$\\
$\Leftrightarrow$ $dRe$, $eRd$\\
ent. $e=d$ (Por (2))\\\

$xR'y$, $yR'z$ ent $xR'z$\\
$h(x)Rh(y)$, $h(y)Rh(z)$\\
$\Leftrightarrow$ $dRe$, $eRk$ (Por (3))\\
$dRk$ $\Leftrightarrow$ $h(x)Rh(z)$\\
$\Leftrightarrow$ $xR'z$\\
Por lo tanto $(x, R')$
\end{proof}




\begin{prob}[Ejerc\'icio 7.4.17] %[Fuente, capítulo, ejercicio.]
Sean $(A, \preceq)$ y $(B, \preceq)$ COPOs y sea $f: A \longrightarrow B$ un isomorfismo de orden. Probar que si $\preceq$ es un orden total, entonces tambi\'en es $\preceq'$.\\\\
\end{prob}
\begin{proof}
\end{proof}


\begin{prob}[Ejerc\'icio 4.2.1] %[Fuente, capítulo, ejercicio.]
Encuentra el rango de cada una de las siguientes funciones.\\
(1) Sea $f: \Re \longrightarrow \Re$ definida por $f(x) = x^6 - 5$ para toda $x \in \Re$.\\
(2) Sea $g: \Re \longrightarrow \Re$ definida por $g(x) = x^3 - x^2$ para toda $x \in \Re$\\ 
(3) Sea $h: \Re \longrightarrow (0, \infty)$ definida por $h(x) = e^(x -1) + 3$ para toda $x \in \Re$\\
(4) Sea $p: \Re \longrightarrow \Re$ definida por $p(x) = \surd x^4 + 5$ para toda $x \in \Re$\\
(5) Sea $q: \Re \longrightarrow [-10,10]$ definida por $q(x) = \sin x + \cos x$ para toda $x \in \Re$\\
\end{prob}

\begin{proof}
$f: \Re \longrightarrow \Re$\\
(1) $y = x^6 - 5$\\
$y + 5 = x^6$, $x^6 - 5 < -5$\\
$^6\surd y+5 = x$, $x^6<0$\\
$y+5 \geq 0$\\
$y \geq -5$\\
$R = \{ y \in \Re \mid y \geq -5 \}$\\
$z \in [-5, \infty)$\\\

(2) $f: \Re \longrightarrow \Re$\\
$y = x^3 - x^2$\\
$y = x^2 (x-1)$\\
$y \in \Re$\\\

(3) $f: \Re \longrightarrow (0, \infty)$\\
$y = e(^x+1) + 3$, $(3, \infty)$\\
$y - 3 = e(^x-1)$\\
$ln(y-3) = ln e(^x-1)$\\
$ln (y-3) = x-1$\\\

(4) $f: \Re \longrightarrow \Re$\\
$y^2 = x^4 + 5$\\
$y^2 - 5 = x^4$\\
$\surd y^2 -5 = x$\\
$y^2 -5 \geq 0$\\
$y^2 \geq 5$\\
$y^2 \geq \surd 5$
\end{proof}




\begin{prob}[Ejerc\'icio 4.2.3] %[Fuente, capítulo, ejercicio.]
\end{prob}

\begin{proof}

\end{proof}



\begin{prob}[Ejerc\'icio 4.2.5] %[Fuente, capítulo, ejercicio.]
Sean $X$ y $Y$ conjuntos, sean $A \subseteq C$ y $B \subseteq Y$ subconjuntos y sean $\Pi: X \times Y \longrightarrow X$ y $\Pi: X \times Y \longrightarrow Y$  mapas proyectados como los definidos en la secci\'on 4.1.\\
(1) Probar que $(\Pi1)^-1 (A) = A \times Y$ y $(\Pi2)^-1 (B)$ $= X \times B$\\
(2) Probar que $(\Pi1)^-1 (A) \cap (\Pi2)^-1 (B) = A \times B$\\
(3) Sea $P \subseteq X \times Y$. Es $\Pi1(P) \times \Pi2(P) = P$? Da una prueba o un contraejemplo.\\
\end{prob}

\begin{proof}

\end{proof}



\begin{prob}[Ejerc\'icio 4.2.10] %[Fuente, capítulo, ejercicio.]
(1) Encuentra un ejemplo de una funci\'on $f: A \longrightarrow B$ y subconjuntos $P,Q \subseteq A$ tal que $P \subseteq \neq Q$, pero que $f(P) = f(Q)$.\\
(2) Encuentra un ejemplo de una funci\'on $g: C \longrightarrow D$ y subconjuntos $S,T \subseteq D$ tal que $S \subseteq \neq T$, pero que $g^-1(S) = g^-1(T)$.
\end{prob}

\begin{proof}
(2) $C, D = \mathbb{R}$,\  $g^{-1}(S) = g^{-1}(T) f(x)=x^{2}$

$S: \mathbb{N}$\\ 
$\mathbb{N}, \mathbb{R} \subseteq \mathbb{R}$

$T=Z$\\
$\mathbb{N} \subsetneq  \mathbb{R}$

$g^{-1}(S) = g^{-1}(T)$

$x \in \mathbb{R}$ tales que $f(x)=y$, y $\in \mathbb{N}$

$x = dy$ para alg\'una $y \in \mathbb{N}$

$x^{2} = Z$
$x= \sqrt{z}$\\
$(-\infty, 0) = \emptyset$

\end{proof}



\begin{prob}[Ejerc\'icio 4.2.11] %[Fuente, capítulo, ejercicio.]
Sean $A$ y $B$ conjuntos, sean $P,Q \subseteq A$ subconjuntos y sea $f: A \longrightarrow B$ una funci\'on.\\
(1) Probar que $f(P) - f(Q) \subseteq f(P - Q)$.\\
(2) Es necesariamente el caso que $f(P - Q) \subseteq f(P) - f(Q)$? Da una prueba o un contraejemplo.
\end{prob}

\begin{proof}
1) Sea $y \in f(P) - f(Q)$\\
$y= \{f(x): x \in P\} - \{f(x): x \in Q\}$\\
$y = f(x)$ tal que $x \in P$ y $x \notin Q$\\
$y = f(x)$ para alg\'un $x \in P - Q$\\
$\to$ $y \in \{f(x): x \in P-Q\} = f(P - Q)$\\
Por lo tanto $y \in f(P-Q)$ $\to f(P) - f(Q) \subseteq f(P - Q)$\\\

2) $P=\{1, 2, 3\}$ , $Q=\{-1, -2, -3\}$\\
$f(x)= x^\{2\}$\\
$f(P-Q)= \{-3,-2,-3, 1, 2, 3\}$\\
$f(P)-f(Q)=\{1, 2, 3\}$\\
Por lo tanto $f(P-Q) \subset f(P) - f(Q)$
\end{proof}



\begin{prob}[Ejerc\'icio 4.2.12] %[Fuente, capítulo, ejercicio.]
Sean $A$ y $B$ conjuntos, sean $C,D \subseteq B$ subconjuntos y sea $f: A \longrightarrow B$ una funci\'on. Probar que $f^-1(D - C) = f^-1 (D) - f^-1(C)$.
\end{prob}

\begin{proof}
$f: A \longrightarrow B$\\ $C,D \subseteq B$\\
PD $f(^-1) (D - C) = f(^-1)(D) - f(^-1)(C)$\\\

$(\subseteq)$\\
Sea $a \in f(^-1) (D - C) \Longrightarrow f(a) \in (D-C)$\\
$\Longrightarrow f(a) \in D$ y $f(a) \notin C$\\
$\Longrightarrow a \in f(^-1) (D) - f(^-1) (C)$\\\

$(\supseteq)$\\
Sea $a \in f(^-1) (D) - f(^-1) (C)$\\
$\Longrightarrow a \in f(^-1) (D)$ y $a \notin f(^-1) (C)$\\
$\Longrightarrow f(a) \in D$ y $f(a) \notin C$\\
$\Longrightarrow f(a) \in D - C$\\
$\Longrightarrow a \in f(^-1)(D-C)$.
\end{proof}


\begin{prob}[Ejerc\'icio 4.3.3] %[Fuente, capítulo, ejercicio.]
Sea $f, g: \Re \longrightarrow \Re$ definida por\\
$f(x) = ( 1 - 2x, si x \geq 0),  f(x)(|x|, si x < 0)$\\
$g(x)=( 3x, si x \geq 0)  g(x)(x-1, si x < 0)$\\
Encuentra $fog$ y $gof$
\end{prob}

\begin{proof}
Cuando $x \geq 0$\\
$f(x) = 1-2x$\\
$g(x) = 3x$\\
$fog = f(g(x)) = 1 - 2(3x)$\\
$gof = g(f(x)) = 3(1-2x)$\\\

Cuando $ x < 0$\\
$fog = f(g(x)) = |x-1|$\\
$gof = g(f(x)) = |x|-1$\\
$fog = ( 1-6x \mid x \geq 0$, $|x-1| x < 0)$\\
$gof = (3-6x \mid x \geq 0$, $|x|-1 x < 0)$.

\end{proof}




\begin{prob}[Ejerc\'icio 4.3.5] %[Fuente, capítulo, ejercicio.]
Sean A y B conjuntos, sea $U \subseteq A$ y $V \subseteq C$ subconjuntos, y sea $f: A \longrightarrow B$ y  $g: B \longrightarrow C$ funciones. Probar que:\\
$(gof) (U) = g(f(U))$ y $(gof)^{-1}(V) = f^{-1}(g^{-1}(V))$.
\end{prob}

\begin{proof}
P.D. $(gof)(u) = g(f(u))$\\ 
P.D $y \in (gof)(u) \leftrightarrow y \in g(f(u))$\\\

Sea $y \in (gof)(u)$\\
$y \in (gof)(u) \leftrightarrow y \in (gof)(u): u \in U$\\

$y = (gof)(u)$ para alg\'una $u \in U$\\
$\leftrightarrow$ $y = g(F(u))$ para alguna $u \in U$\\
$\leftrightarrow$ $y = g(z) con z = f(u)$ para alguna $u \in U$��
$\leftrightarrow$ $y = g(z)$ para alguna $z \in f(u)$\\
$\leftrightarrow$ $y \in g(f(u))$\\
Por lo tanto $(gof)(u) = g(f(u))$.\\\

$V \subseteq C$\\
$(gof)-1(V) = f-1(g-1(V))$\\
Sea $y \in (gof)(V) = (x \in A: (gof)(x) \in V)$\\
$(gof)(y) \in V \leftrightarrow g(f(y)) \in V \leftrightarrow f(x) \in g-1(v)$\\
$f-1(g-1(u)) = (x \in A: f(x) \in g-1(v))$\\
$\leftrightarrow$ $y \in f-1(g-1(v))$

\end{proof}




\begin{prob}[Ejerc\'icio 4.3.6] %[Fuente, capítulo, ejercicio.]
Sean $A,B$ y $C$ conjuntos, y sean y sea $f: A \longrightarrow B$ y  $g: B \longrightarrow C$ funciones. Suponer que $f$ y $g$ tienen inversas. Probar que $gof$ tienen inversa, y que $(gof)^{-1} = f^{-1}og^{-1}$
\end{prob}

\begin{proof}
Sean $f: A \longrightarrow B$, $g:B \to C$ funciones invertibles\\
Como $f$ y $g$ tienen inversa\\
entonces llamemos $f^{-1}:B \to A g^{-1}:C \to B$ inversa.\\
PD $gof$ tiene inversa\\
Sea $h =f^{-1}og^{-1}, h:C \to A$\\
PD $h$ es inversa de $gof$\\
PD $h(gof) = (gof)oh=Id$  con $c \in C$\\
$[ho(gof)](c) = [(f-1og-1)o(gof)](c)$\\
$ \longrightarrow [f^{-1}o(g^{-1}o(gof))](c)$\\
$\longrightarrow [f^{-1}o(g^{-1}og)of)](c)$\\
$\longrightarrow (f^{-1}o(Iof))(c)$\\
$\longrightarrow (f^{-1}of)(c) = I(c) = c$\\
Por lo tanto $h$ es inversa $(goh)$\\
$(goh)^{-1} = h$\\
$\Leftrightarrow$ $(goh)^{-1} = f^{-1}og^{-1}$.
\end{proof}





\begin{prob}[Ejerc\'icio 4.3.7] %[Fuente, capítulo, ejercicio.]
\end{prob}

\begin{proof}
(1) $g:[0, \infty) \to \mathbb{R}$ tal que $hog = Id_{[0, \infty)}$\\
$g_{1}(x) = -x$\\
$x \in[0, \infty)$\\
$h(g_{1}(x)) =h(-x) = -(-x) = x$\\
$g_{2}(x)=x$\\
$h(g_{2}(x)) = h(x) = x$\\\

(2) $f: \mathbb{R} \to [1, \infty)$\\
$f(x) = {e^{x}}^{2}$\\
$g_{1}, g_{2}: [1, \infty) \to \mathbb{R}$\\
$fog_{1} = Id_{(1, \infty)}$\\
$fog_{2} = Id[1, \infty)$\\
$g(x) = log(x^{2})$\\
$g(x) = \sqrt{log(x)}$\\
$fog(x)= e^{log(x)} = x$.

\end{proof}





\begin{prob}[Ejerc\'icio 4.3.8] %[Fuente, capítulo, ejercicio.]
\end{prob}

\begin{proof}
(1) $f(x) = x^{3} + 4$\\
Encontrar a $g$ tal que $gof(x) = I$\\
$y=x^{3} + 4 \to \sqrt[3]{x-4} = x$\\
$g(x) = \sqrt[3]{x-4}$\\
$gof(x) = g(x^{3} + 4) = \sqrt[3]{(x^{3} + 4)-4} = x = I$\\\

2) $g: \mathbb{R} \to \mathbb{R}$\\
$g(x) = e^{x} \forall x \in \mathbb{R}$\\
$h(x) = logx$
$hog = loge^{x} = x$

\end{proof}





\begin{prob}[Ejerc\'icio 4.4.7] %[Fuente, capítulo, ejercicio.]
\end{prob}

\begin{proof}
P.D. $f(X) = f(Y) \to x = y$\\
$x \neq y \to f(x) \neq f(y)$\\
$Sup. f(x) = f(y) \to A - X = A - Y$\\
$Si a \in A - X \leftrightarrow a \in A - Y$\\
$a \in A$ y $a \notin X \leftrightarrow a \in A$ y $a \notin Y$\\
$a \in A, a \in X^{c} \leftrightarrow a \in A, a \in Y^{c}$\\
como $X^{c} \subseteq A$, $Y^{c} \subseteq A$\\
$a \in A \cap X^{c} \leftrightarrow a \in A \cap Y^{c}$\\
$a \in X^{c} \leftrightarrow a \in Y^{c}$\\
$a \notin X \leftrightarrow a \notin Y$\\
$\neg(a \in X) \leftrightarrow \neg(a \notin Y) \to a \in X \leftrightarrow a \in Y$\\
entonces $x = y$\\
$\Longrightarrow$ es inyectiva

\end{proof}





\end{document}




