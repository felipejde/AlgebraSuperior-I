\documentclass[12pt,oneside]{article}
\usepackage[T1]{fontenc}
\usepackage{latexsym}
\usepackage[activeacute,spanish]{babel}
\usepackage{amsfonts}
\usepackage{amsmath}
\usepackage{amssymb}
\usepackage{amsthm}
\usepackage{graphicx}
\usepackage[all]{xy}
\usepackage{tikz}
\usepackage[normalem]{ulem}
\usepackage{cancel}
\usepackage{soul}
\usepackage[retainorgcmds]{IEEEtrantools}
\usepackage{mathrsfs}
\usepackage{makeidx}
\newtheorem{prob}{Problema}
\addtolength{\hoffset}{-2cm}
\addtolength{\textwidth}{4cm}
\addtolength{\voffset}{-2.5cm}
\addtolength{\textheight}{5cm}
\pagestyle{empty}


\begin{document}


\begin{flushright}
{\large\textbf{Cova} \textnormal{Pacheco}, \textnormal{Felipe de Jes\'us}}
\end{flushright}



\begin{prob}[Ejerc\'icio 6.3.1] %[Fuente, capítulo, ejercicio.]
Probar que cada una de las siguientes f\'ormulas sostiene para todo $n \in \mathbb{N}$\\
(1) $1 + 3 + 5 + ... + (2n - 1) = n^2$.\\
(2) $1^2 + 2^2 + ... + n^2 = (n(n+1)(2n+1))/6$.\\
(3) $1^3 + 2^3 + ... + n^3 = (n^2(n+1)^2)/4$.\\
(4) $1^3 + 3^3 + ... + (2n-1)^3 = (n^2(2n^2 - 1)).$\\
(5) $1*2 + 2*3 + ... +  n(n+1) = (n(n+1)(n+2))/3$.\\
(6) $(1/1*2) + (1/2*3) + ... + (1/n(n+1)) = (n/n+1)$.
\end{prob}

\begin{proof}
(A) $1+3+5+...+(2n-1) = n^2$\\
Base: $n = 1 \Longrightarrow 1=1^2 = 1$\\
H.I.: $n = k \Longrightarrow 1+3+5+...+(2k-1)=k^2$\\
P.D.: $n = k+1$, $1+3+...+(2(k+1)-1) = (k+1)^2$\\
$1+3+...+(2k-1)+(2(k+1)-1) = k^2 + (2(k+1)-1)$\\
$k^2 + 2k +2-1 = k^2 + 2k+1 = (k+1)^2$\\\
(B) $1^2 + 2^2 +...+n^2 = (n(n+1)(2n+1)) \div 6$\\
Base: $n=1$, $1^2 = 1=1 \div 6 = (1(1+1)(2(1)+1)) \div 6$\\
H.I.: $n=k$, $1^2 + 2^2 +...+k^2 = (k(k+1)(2k+1)) \div 6$\\
P.D.: $n=k+1$, $1^2 + 2^2 +...+(k+1)^2 = ((n+1)(n+2)(2n+3)) \div 6$\\
$\Longrightarrow ((n(n+1)(2n+1)) \div 6) + (n+1)^2$\\
$= ((n(n+1)(2n+1)) + 6(n+1)^2) \div 6$\\
$(n+1) ((n(2n+1) + 6(n+1)) \div 6)$\\
$((n+1)(2n^2 + n + 6n + 6)) \div 6$\\
$((n+1)(2n^2+7n+6)) \div 6$\\
$((n+1)(n+2)(2n+3)) \div 6$
\end{proof}

\begin{prob}[Ejerc\'icio 6.3.2] %[Fuente, capítulo, ejercicio.]
Probar que $1 + 2n \leq 3^n$ para todo $n \in \mathbb{N}$.
\end{prob}

\begin{proof}
Base: $1+2(1) \leq 3'$\\
$3 \leq 3$\\
P.D.:
$1+2(n+1) \geq 3^{n+1}$\\
$1+2(n+1)-3^{n+1} = 1+2n+2-3^n�3$\\
$= 2n + 1 - 3^n(2+1)+2$\\
$= 2n + 1 - 3^n*2-3^n+2$\\
$= 2n + 1 -3^n \leq 0 + 2(-3^n + 1) \leq 0$\\
$1+2(n+1)-3^{n+1} \leq 0$\\
$1+2(n+1) \leq 3^{n+1}$
\end{proof}



\begin{prob}[Ejerc\'icio 6.3.3] %[Fuente, capítulo, ejercicio.]
Sean $a, b \in N$. Probar que $a^n - b^n$ es divisible por $a-b$ para todo $n \in \mathbb{N}$. 
\end{prob}

\begin{proof}
Base: $n = 1$\\
P.D.: $a-b \mid a-b$ ya que $(a-b) = 1�(a-b)$\\
Sup. hasta $k$\\
H.I.: $(a-b) \mid a^k - b^{k}$\\
P.D.: $(a-b) \mid a^{k+1} - b^{k+1}$\\
$a^{k+1} - b^{k+1}$\\
$(a^{k} - b^{k}) (a+b) = a^{k+1} + a^kb - b^ka - b^{k+1}$\\
$= a^{k+1} - b^{k+1} + a^kb - ab^k$\\
$= a^{k+1} - b^{k+1} + ab(a^{k-1} - b^{k-1})$\\
$(a^{k} - b^{k}) (a+b) - ab(a^{k+1} - b^{k+1}) = a^{k+1} - b^{k+1}$\\
$(a-b) \mid a^k - b^k \Longrightarrow  a^k - b^k = h (a-b)$\\
$a-b \mid a^{k-1} - b^{k-1} \Longrightarrow a^{k-1} - b^{k-1} = k (a-b)$\\
$h(a-b)(a+b) - ab k(a-b) = a^{k+1} - b{k+1}$\\
$(a-b) [h(a+b) - abk] = a^{k+1} - b^{k+1}$\\
$\Longrightarrow a-b \mid a^{k+1} - b^{k+1}$
\end{proof}



\begin{prob}[Ejerc\'icio 6.3.4] %[Fuente, capítulo, ejercicio.]
Sea $f: \mathbb{N} \longrightarrow \mathbb{N}$ una funci\'on. Suponemos que $f(n) < f(n+1)$ para todo $n \in \mathbb{N}$. Probar que $f(n) \geq n$ para todo $n \in \mathbb{N}$.
\end{prob}

\begin{proof}
Sea $k \in \mathbb{N}$\\
$\Longrightarrow f: \mathbb{N} \longrightarrow \mathbb{N} \Longrightarrow f(k) \in \mathbb{N}$\\
$\Longrightarrow f(k) \geq 0$, en particular $k=0$\\
$f(0) \geq 0$ Base de inducci\'on\\
Sup. $f(k) \geq k$ P.D.: $f(k+1) \geq k+1$\\
Por Hip. $f(n) < f(n+1)$\\
$\Longrightarrow k \leq f(k) < f(k+1) \Longrightarrow k < f(k+1)$\\
$\Longrightarrow k+1 \leq f(k+1)$

\end{proof}



\begin{prob}[Ejerc\'icio 6.3.6] %[Fuente, capítulo, ejercicio.]
Para cada valor de $n \in \mathbb{N}$ se contiene la desigualdad $n^2 - 9n+19 > 0$? Pru\'ebalo por inducci\'on. 
\end{prob}

\begin{proof}
$n^2 - 9n + 19 > 0$, $n=0,1,2$\\
Base: $n = 6$\\
$6^2 - 9(6) + 19 = 36 - 54 +19 > 0$\\
Sup. $k^2 - 9k + 19 > 0$\\
P.D.: $(k+1)^2 - 9(k+1) + 19 > 0$\\
$(k+1)^2 - 9(k+1) + 19 = k^2 + 2k + 1 - 9k - 9 + 19$\\
$= (k^2 - 9k + 19) + (2k+1-9)$\\
$= (k^2 - 9k + 19) + (2k - 8)$\\
H.I. \qquad como $k \geq 6$\\ 
$> 0 + (2k-8)$ \qquad $2k \geq 12$\\
$> 0 + 0 = 0$ \qquad $2k-8 \geq 4 > 0$\\
Por lo tanto $k^2 - 9k + 19 > 0$
\end{proof}
\end{document}




