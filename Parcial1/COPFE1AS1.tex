\documentclass[12pt,oneside]{article}
\usepackage[T1]{fontenc}
\usepackage{latexsym}
\usepackage[activeacute,spanish]{babel}
\usepackage{amsfonts}
\usepackage{amsmath}
\usepackage{amssymb}
\usepackage{amsthm}
\usepackage{graphicx}
\usepackage[all]{xy}
\usepackage{tikz}
\usepackage[normalem]{ulem}
\usepackage{cancel}
\usepackage{soul}
\usepackage[retainorgcmds]{IEEEtrantools}
\usepackage{mathrsfs}
\usepackage{makeidx}
\newtheorem{prob}{Problema}
\addtolength{\hoffset}{-2cm}
\addtolength{\textwidth}{4cm}
\addtolength{\voffset}{-2.5cm}
\addtolength{\textheight}{5cm}
\pagestyle{empty}


\begin{document}


\begin{flushright}
{\large\textbf{Ex\'amen 1} \textnormal{Cova Pacheco Felipe de Jes\'us}, \textnormal{Jueves 25 de febrero de 2016}}
\end{flushright}


\begin{prob} %[Fuente, capítulo, ejercicio.]
Sean $A$ y $B$ conjuntos, demuestre que:\\\
a) $\wp (A) \cup \wp (B) \subseteq \wp (A \cup B)$ y es estricta si $A \nsubseteq B$ o $B \nsubseteq A$.\\
b) $\wp(A - B) - \{ \emptyset \} \subseteq \wp(A) - \wp(B)$ y es estricta si $A \nsubseteq B$
\end{prob}

\begin{proof}
a) Sea $x \in \wp(A) \cup \wp(B)$\\
$x \in \wp(A)$ \'o $x \in \wp(B)$\\
entonces $x \subseteq A$ \'o $x \subseteq B$\\
entonces $x \subseteq A \cup B$\\
$x \in \wp(A \cup B)$\\
Por lo tanto $\wp(A) \cup \wp(B) \subseteq \wp(A \cup B)$\\\

b) Sea $x \in \wp(A-B) - \{ \emptyset \}$\\
ent. $x \in \wp(A-B)$ y $x \notin \{ \emptyset \}$ entonces $x \neq \emptyset$\\
ent. $x \subseteq A-B$\\
ent. $x \subseteq A$ y como $x \neq \emptyset$ puede pasar que $x \nsubseteq B$\\
$x \in \wp(A)$ y $w \notin \wp(B)$\\
$x \in \wp(A) - \wp(B)$\\
Por lo tanto $\wp(A-B) - \{ \emptyset \} \subseteq \wp(A) - \wp(B) $\\\\
\end{proof}



\begin{prob} %[Fuente, capítulo, ejercicio.]
Para n\'umeros reales $a, b$ y $c$ se sabe que $a-(b-c) = (a-b)+c$. Considere $A, B, C$ conjuntos, encuentre y demuestre una f\'ormula, equivalente a la de n\'umeros reales, para\\
    $A-(B-C)$.
\end{prob}

\begin{proof}
$\subseteq )$\\
Sea $x \in A-(B-C)$\\
$x \in A$ y $x \notin (B-C)$\\
ent. \'o $x \in B y x \in C$ \'o $x \notin B y x \notin C$ \'o $x \notin B y x \in C$\\
Caso 1:\\
$x \in A$ y $x \in B$ y $x \in C$\\
ent. $x \notin A-B$\\
$x \notin (A-B) \cup C$\\
Caso 2:\\
$x \in A$ y $x \notin B$\\
$x \in (A-B) \cup C$\\\

$\supseteq )$\\
Sea $x \in (A-B) \cup C$\\
$x \in (A-B)$ \'o $x \in C$\\
$\bullet$ Si $x \in A-B$\\
entonces $x \in A$ y $x \notin B$\\
Como $x \notin B$ entonces $x \notin B-C$\\
entonces $x \in A$ y $x \notin (B-C)$ entonces $x \in A - (B-C)$\\
$\bullet$ Si $x \in C$\\
$\bullet$ Si $x \in B$ ent $x \notin B-C$\\
$\bullet$ Si $x \in A$ ent. $x \in A-(B-C)$\\
$\bullet$ Si $x \notin A$ ent. $x \notin A-(B-C)$\\
$\bullet$ Si $x \notin B$ ent. $x \notin B-C$\\
$\bullet$ Si $x \in A$ ent. $x \in A-(B-C)$\\
$\bullet$ Si $x \notin A$ ent. $x \notin A-(B-C)$\\\\
\end{proof}


\begin{prob} %[Fuente, capítulo, ejercicio.]
Pregunta de rescate: En los \'ultimos 153 a�os, el gobierno mexicano ha mandado dos notas diplom\'aticas a El Vaticano, diga cuales fueron los sucesos que provocaron dichas notas.
\end{prob}

\begin{proof}
Gobierno de Benito Ju\'arez: Extinci\'on de las comunidades religiosas en M\'exico.\\
Gobierno de Enrique Pe�a Nieto: No a la mexicanizaci\'on en Argentina.
\end{proof}


\end{document}




