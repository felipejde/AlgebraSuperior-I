\documentclass[12pt,oneside]{article}
\usepackage[T1]{fontenc}
\usepackage{latexsym}
\usepackage[activeacute,spanish]{babel}
\usepackage{amsfonts}
\usepackage{amsmath}
\usepackage{amssymb}
\usepackage{amsthm}
\usepackage{graphicx}
\usepackage[all]{xy}
\usepackage{tikz}
\usepackage[normalem]{ulem}
\usepackage{cancel}
\usepackage{soul}
\usepackage[retainorgcmds]{IEEEtrantools}
\usepackage{mathrsfs}
\usepackage{makeidx}
\newtheorem{prob}{Problema}
\addtolength{\hoffset}{-2cm}
\addtolength{\textwidth}{4cm}
\addtolength{\voffset}{-2.5cm}
\addtolength{\textheight}{5cm}
\pagestyle{empty}


\begin{document}

\begin{flushleft}
{\large\textbf{Ex\'amen 2}}
\end{flushleft}


\begin{flushright}
{\large\textbf{Cova} \textnormal{Pacheco}, \textnormal{Felipe de Jes\'us}}
\end{flushright}



\begin{prob} %[Fuente, capítulo, ejercicio.]
Considere $f \subseteq \mathbb{R}$ x $\mathbb{R}$ dada por\\
$(x,y) \in f$ si y s\'olo si $y = x^2$ si $x \ge 0$, $y = x$ si $x \le 0$\\
Conteste argumentando ampliamente las siguientes preguntas:\\
i. $f$ es reflexiva?\\\
ii. $f$ es sim\'etrica?\\\
iii. $f$ es transitiva?\\\
iv. $f$ es de equivalencia?\\\
v. Calcule $f[1], [1]f, f[-2]$ y $[-2]f$.\\\
vi. $f$ es funci\'on?\\\
vii. $f$ es inyectiva?\\\
viii. $f$ es suprayectiva?\\\
ix. $f$ es biyectiva?\\\
x. Calcule $f[[-1,1]], f^{-1}[f[[-1,1]]], f^{-1}[[-1,1]], f[f[^{-1}[-1,1]]]$ considerando al\\
 intervalo $[-1,1]$ como uni\'on de los intervalos $[-1,0]$ y $[0,1]$.\\\
\end{prob}

\begin{proof} i) Reflexiva:\\
La funci\'on no es reflexiva, ya que por ejemplo $f(2) = 2^2 = 4$ y $f(4) = 4^2 = 16$\\
Por lo tanto no es reflexiva.\\
\end{proof}


\begin{proof} ii) Sim\'etrica:\\
Para toda $x, f(-x) = f^{-1}(x)$, entonces\\
$f(-2) = -2$ y $f(2) = 2^2 = 4$\\
\end{proof}


\begin{proof} iii) Transitiva:\\
$f(2*3) = (2*3)^2 = 6^2 = 36$\\
$f(3*1) = (3*1)^2 = 3^2 = 9$, ent. $f(2*1) = (2*1)^2 = 4$\\
i.e. la funci\'on no cumple  la propiedad de la transitividad.\\
Como por ejemplo\\
$f(2) = 2^2 = 4$ y $f(4) = 4^2 = 16$\\
$f(2) \neq f(16)$\\
\end{proof}


\begin{proof} iv) Equivalencia:\\
No es de equivalencia, ya que no cumple con ninguna de las tres propiedades, reflexiva, sim\'etrica y transitiva.\\
\end{proof}


\begin{proof} vi)\\
Si es funci\'on, porque es una relaci\'on y cumple que es inyectiva\\
porque para toda $x \in \mathbb{R}, f(x) = y$ y $y \in \mathbb{R}$\\
\end{proof}


\begin{proof} vii)\\
Es inyectiva, porque para toda $x \in \mathbb{R}$ existe un \'unico $y \in \mathbb{R}$\\
que es diferente de $f(x) = y$\\
y por lo tanto $f(x_0) = f(x_1)$ y a cada valor de $y$ le corresponde uno y s\'olo un valor de $x$\\
\end{proof}


\begin{proof} viii)\\
Si es suprayectiva, porque el rango de la funci\'on es igual al dominio. En otras palabras, todo $y$ tiene un valor en el dominio.\\
\end{proof}


\begin{proof} ix)\\
Si es biyectiva porque cumple que es inyectiva y suprayectiva.\\
\end{proof}

\begin{prob} %[Fuente, capítulo, ejercicio.]
Pregunta de rescate:\\
Con la reforma pol\'itica de la Ciudad de M\'exico, los habitantes podr\'an elegir a sus autoridades municipales, antes delegacionales. Ser\'a esta la primera vez que los habitantes de la Ciudad de M\'exico podr\'an elegir a sus autoridades municipales?
\end{prob}

\begin{proof}
No, no ser\'a la primera vez, hace veinte a�os ya suced\'ia.
\end{proof}



\end{document}