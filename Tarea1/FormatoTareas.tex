\documentclass[12pt,oneside]{article}
\usepackage[T1]{fontenc}
\usepackage{latexsym}
\usepackage[activeacute,spanish]{babel}
\usepackage{amsfonts}
\usepackage{amsmath}
\usepackage{amssymb}
\usepackage{amsthm}
\usepackage{graphicx}
\usepackage[all]{xy}
\usepackage{tikz}
\usepackage[normalem]{ulem}
\usepackage{cancel}
\usepackage{soul}
\usepackage[retainorgcmds]{IEEEtrantools}
\usepackage{mathrsfs}
\usepackage{makeidx}
\newtheorem{prob}{Problema}
\addtolength{\hoffset}{-2cm}
\addtolength{\textwidth}{4cm}
\addtolength{\voffset}{-2.5cm}
\addtolength{\textheight}{5cm}
\pagestyle{empty}


\begin{document}


\begin{flushright}
{\large\textbf{Cova} \textnormal{Pacheco}, \textnormal{Felipe de Jes\'us}}
\end{flushright}



\begin{prob}[Ejerc\'icio 7.4.2] %[Fuente, capítulo, ejercicio.]
Es cada una de las siguientes relaciones antisim\'etrica, un orden parcial y/o un orden total?\\
$C =  \{n \in \mathbb{Z}$ | existe k $\in \mathbb{Z}$ tal que $n = k\};$\\
$E =  \{n \in \mathbb{Z}$ | existe k $\in \mathbb{Z}$ tal que $n = 2k\};$\\
$P =  \{n \in \mathbb{Z}$ | $n$ es un n\'umero primo\};\\
$N =  \{n \in \mathbb{Z}$ | existe k $\in \mathbb{Z}$ tal que $n = k\};$\\
$S =  \{n \in \mathbb{Z}$ | existe k $\in \mathbb{Z}$ tal que $n = 6k\};$\\
$D =  \{n \in \mathbb{Z}$ | existe k $\in \mathbb{Z}$ tal que $n = k - 5\};$\\
$B =  \{n \in \mathbb{Z}$ | n es no negativo\};
\end{prob}

\begin{proof}
\end{proof}

\begin{prob}[Ejerc\'icio 3.2.9] %[Fuente, capítulo, ejercicio.]
Encuentra conjuntos $A$ y $B$ tal que $A \in B$ y $A \subseteq B$.
\end{prob}

\begin{proof}
Sea $A = \{\emptyset\}$ y $B = \{\emptyset, \{\emptyset\}\}$\\
$A$ es un conjunto que contiene al conjunto vac�o\\
$B$ tiene dos elementos, el conjunto vac�o y el conjunto que contiene al conjunto vac�o
as� que como todos los elementos de $A$ est\'an en $B$, se cumple que $A \subset B$
y como $B$ contiene al conjunto que contiene al conjunto vac�o, se concluye que $A \in B$\\
\end{proof}



\begin{prob}[Ejerc\'icio 3.2.10] %[Fuente, capítulo, ejercicio.]
Sean $A, B$ y $C$ conjuntos. Suponemos que $A \subseteq B$ y $B \subseteq C$. Probar que $A = B = C$.
\end{prob}

\begin{proof}
Por definici\'on de subconjunto:\\
$A \subseteq B$ para toda $x \in A, x \in B$ (1)\\
$B \subseteq C$ para toda $x \in B, x \in C$ (2)\\
De 1 y 2 tenemos que para toda $x \in A, x \in C$  (3) es decir $A \subseteq C$\\
Como $C \subseteq A$ Para toda $x \in C, x \in A$ (4)\\ 
de 3 y 4 tenemos que $A = C$\\
Como $A = C$ 2 puede reescribirse como\\
$B \subseteq A$ Para toda $x \in B, x \in A$ (5)\\
entonces de 1 y 5 tenemos que $A = B$ Por lo tanto $A = B = C$\\
\end{proof}



\begin{prob}[Ejerc\'icio 3.2.11] %[Fuente, capítulo, ejercicio.]
Sean $A$ y $B$ conjuntos. Probar que no es posible que $A \subsetneq B$ y $B \subseteq A$ ambas sean correctas.
\end{prob}

\begin{proof}
Por definici\'on si $A \subsetneq?? B$ Para toda $a \in A, a \in B$ pero existe al menos una $b \in B, b \notin A$ si $B \subseteq A$ para toda $b \in B, b \in A$ lo cual es una contradicci\'on con nuestro enunciado anterior,
ya que solo sucede que $b \notin A$ o $b \in A$ pero no ambas a la vez.\\ 
\end{proof}




\begin{prob}[Ejerc\'icio 3.2.12] %[Fuente, capítulo, ejercicio.]
Sean $A y B$ cualesquiera dos conjuntos. Es correcto que uno de $A \subseteq B$ o $A = B$ o $A \supseteq B$ deben ser verdad?. Da una prueba o un contraejemplo.
\end{prob}

\begin{proof}
Falso\\
Contra Ejemplo: Sea $A = \{3,4,5\}$ y $B = \{8,4,1\}$
$A \nsubseteq B$ ya que no todos los elementos de $A$ est\'an en $B$\\
Si no se dio la contenci\'on, mucho menos la igualdad, $A \neq B$\\
$A \nsubseteq B$ ya que no todos los elementos de $B$ est\'an en $A$\\
\end{proof}




\begin{prob}[Ejerc\'icio 3.2.13] %[Fuente, capítulo, ejercicio.]
Sea $A = \{x,y,z,w\}$. Enlista todos los elementos en  $\wp(A)$.
\end{prob}

\begin{proof}
$A = \{x,y,z,w\}$ ent.\\
$\wp (A)= \{ \emptyset, \{x,y,z,w\}, \{x,y,z\}, \{y,z,w\}, \{z,w,x\}, \{x,y\}, \{x,z\}, \{x,w\},\\
\{y,z\}, \{y,w\}, \{z,w\}, \{x\}, \{y\}, \{z\}, \{w\} \}$.\\
\end{proof}




\begin{prob}[Ejerc\'icio 3.2.14] %[Fuente, capítulo, ejercicio.]
Sean $A$ y $B$ conjuntos. Suponemos que $A \subseteq B$. Probar que $\wp(A) \subseteq \wp(B)$
\end{prob}

\begin{proof}
Por definici\'on $x \in \wp(A)$ si y solo si $x \subseteq A$\\
Si $A \subseteq B$ se cumple que todos los elementos de $A$ est\'an en $B$\\
y como $\wp(B)$ contiene todos los subconjuntos de $B$ quien a su vez contiene todos los elementos\\
de $A$ se cumple que $\wp(A) \subseteq \wp(B)$.\\
\end{proof}


\begin{prob}[Ejerc\'icio 3.2.16] %[Fuente, capítulo, ejercicio.]
??????????????????Cu\'ales de los siguientes son verdaderos y cu\'ales falsos
\end{prob}

\begin{proof}


(1) $\{ \emptyset\} \subseteq G$ para todos conjunto en $G$.\\
	FALSO. Porque el conjunto $G$ no necesariamente tiene al conjunto vac\'io.\\ 
(2)$\emptyset \subseteq G$ para todo conjunto $G$.\\
	VERDADERO. Dem: si $\emptyset \subseteq G$ $\exists x \in \emptyset$ tal que $x \notin G $
	

\end{proof}




\begin{prob}[Ejerc\'icio 3.3.5] %[Fuente, capítulo, ejercicio.]
Dados dos conjuntos $A$ y $B$ los conjuntos $A - B$ y $B - A$ necesariamente disjuntos? Da una prueba o un contraejemplo
\end{prob}

\begin{proof}
Para que los conjuntos $A - B$ y $B - A$ sean disjuntos, no deben tener ning\'un elemento en com\'un.\\
Al no tener elementos en com\'un, la intersecci\'on de ambos conjuntos es el vac\'io\\
$(A - B) \cap (B - A) = \emptyset$\\
$x \in (A - B)$ y las $x \in (B - A)$ por definici\'on de intersecci\'on\\
$x \in A$, $x \notin B$ y las $x \in B$, $\notin A$.\\
no hay ninguna $x$ que cumpla estar y no estar en $A$, lo mismo cumple para $B$ por lo que\\
concluimos que $(A - B) \cap (B - A) = \emptyset$\\
por lo que $A - B$ y $B - A$ son necesariamente disjuntos.\\
\end{proof}



\begin{prob}[Ejerc\'icio 3.3.9] %[Fuente, capítulo, ejercicio.]
Sean $A$ y $B$ conjuntos. Prueba que $(A \cup B) - A = B - (A \cap B)$
\end{prob}

\begin{proof}
$(A \cup B) - A$ son todas las $x$ que est\'an en $A$ o en $B$ que no est\'an en $A$\\ 
Por lo que solo nos quedan las $x$ que est\'an en $B$ que no est\'an en $A$, $B - A$\\
Las $x$ que est\'an tanto en $A$ como en $B$ son las $x \in A$ y $x \in B$
Como $B - A$ son las $x$ que no est\'en en $A$, quitamos de $B$ las $x$ que est\'en en $A$ y en $B$\\ $x \in B$, $x \notin ( x \in A$ y $x \in B )$\\
Por definici\'on de intersecci\'on\\
$x \in B, x \notin ( A \cap B )$\\
Por definici\'on de Diferencia\\
$B - (A \cap B)$\\
Por lo tanto $(A \cup B) - A = B - (A \cap B)$\\
\end{proof}



\begin{prob}[Ejerc\'icio 3.3.10] %[Fuente, capítulo, ejercicio.]
Sean $A$ y $B$ y $C$ conjuntos. Suponemos que $C \subset A \cup B$, y que\\
$C \cap A = \emptyset$. Probar que $C \subseteq B$
\end{prob}

\begin{proof}
Si $x \in C$ y $C \subset A \cup B$\\
$x \in A o B$ por definici\'on de subconjunto y de uni\'on\\
Si ademas $C \cap A = \emptyset$ no existe $x \in C$ y $x \in A$\\
Como para toda $x \in C, x \notin A$\\
tenemos que $\forall x \in C, x \notin A, x \in A$ o $x \in B$\\
Como no existen x que est\'en y no est\'en simult\'aneamente en $A$ nos queda $\forall x \in C, x \in B$ que\\
por definici\'on de subconjunto es $C \subseteq B$\\
\end{proof}



\begin{prob}[Ejerc\'icio 3.3.11] %[Fuente, capítulo, ejercicio.]
Sea $X$ un conjunto, y sea $A, B, C \subseteq X$ son subconjuntos. Supongamos que $A \cap B =
A \cap C$, y que $(X - A) \cap B = (X - A) \cap C$. Probar que $B = C$.
\end{prob}

\begin{proof}
Tenemos $( X - A ) \cap B = ( X - A ) \cap C$\\
como $B \subseteq X, (X - A) \cap B = B - A$\\
como $C \subseteq X, (X - A) \cap C = C - A$\\
$B - A = C - A$\\
$(B - A) \cup (A \cap B) = B$ y $(C - A) \cup (A \cap C) = C$\\
Como $A \cap B = A \cap C y B - A = C - A$\\
$(B - A) \cup (A \cap B) = (C - A) \cup (A \cap C)$\\
Por lo tanto $B = C$.\\
\end{proof}



\begin{prob}[Ejerc\'icio 3.3.12] %[Fuente, capítulo, ejercicio.]
Sean $A, B$ y $C$ conjuntos. Provar que $(A - B) \cap C = (A \cap C) - B = (A \cap C) - (B \cap C)$.	
\end{prob}

\begin{proof}
Sea $x \in (A - B) \cap C$\\
$x \in A - B \rightarrow x \in A$ y $x \notin B$ y $x \in C$\\
como $x \in A$ y $x \in C$ \\
$x \in A \cap C$\\
y como $x \notin B$
Por lo tanto $x \in (A \cap C) - B$\\\\
$x \in (A \cap C) - B$\\
$x \in A$ y $x \in C$ y $x \notin B$\\
$x \in A-B$ y $x \in C$\\
Por lo tanto $x \in (A-B) \cap C$\\\\
$x \in (A \cap C) - (B \cap C)$\\
$x \in (A \cap C)$ y $x \notin (B \cap C)$\\
$x \in A$ y $x \in C$ y $x \notin B$ o $x \notin C$\\
$x \in A$ y $x \in C$ y $x \notin B$\\
Por lo tanto $x \in (A \cap B) - B$\\\\
$x \in (A \cap C) - B$\\
$x \in A \cap C$ y $x \notin B$\\
$x \in A$ y $x \in C$ y $x \notin B \cap C$\\
Por lo tanto $x \in (A \cap C) - (B \cap C)$.


\end{proof}




\begin{prob}[Ejerc\'icio 3.3.16] %[Fuente, capítulo, ejercicio.]
Prueba o encuentra un contraejemplo de la siguiente declaraci\'on. Sean $A, B, C$ conjuntos. Entonces $(A \cup C) - B = (A - B) \cup (C - B)$.
\end{prob}

\begin{proof}
Sea $x ? ((A ? C) ? B)$\\
son las $x \in A$ o las $x \in C$ tal que $x \notin B$\\
las $x \in A$ tal que $x \notin B$, son las $x \in (A - B)$\\
y las $x \in C$ y que $x \notin B$, son las $x \in (C - B)$\\
por lo que las $x$ que est\'an en $(A - B)$ o en $(C  ? B)$\\
son las $x \in (A - B) \cup (C - B)$ por definici\'on de uni\'on
\end{proof}




\begin{prob}[Ejerc\'icio 3.3.5] %[Fuente, capítulo, ejercicio.]
Enlista todos los elementos de cada uno de los siguientes conjuntos.
(1) $\wp(\wp(\emptyset))$.  (2) $\wp(\wp({\emptyset}))$.
\end{prob}

\begin{proof}
(1)
\end{proof}




\begin{prob}[Ejerc\'icio 3.2.16] %[Fuente, capítulo, ejercicio.]
\end{prob}

\begin{proof}
Escriba aqu\'i su segunda demostraci\'on.
\end{proof}




\begin{prob}[Ejerc\'icio] %[Fuente, capítulo, ejercicio.]
\end{prob}

\begin{proof}
Escriba aqu\'i su segunda demostraci\'on.
\end{proof}


\end{document}




